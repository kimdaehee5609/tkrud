%	-------------------------------------------------------------------------------
%
%
%
%
%
%
%
%
%	-------------------------------------------------------------------------------

%\documentclass[10pt,xcolor=pdftex,dvipsnames,table]{beamer}
%\documentclass[10pt,blue,xcolor=pdftex,dvipsnames,table,handout]{beamer}
%\documentclass[14pt,blue,xcolor=pdftex,dvipsnames,table,handout]{beamer}
\documentclass[aspectratio=1610,17pt,xcolor=pdftex,dvipsnames,table,handout]{beamer}

		% Font Size
		%	default font size : 11 pt
		%	8,9,10,11,12,14,17,20
		%
		% 	put frame titles 
		% 		1) 	slideatop
		%		2) 	slide centered
		%
		%	navigation bar
		% 		1)	compress
		%		2)	uncompressed
		%
		%	Color
		%		1) blue
		%		2) red
		%		3) brown
		%		4) black and white	
		%
		%	Output
		%		1)  	[default]	
		%		2)	[handout]		for PDF handouts
		%		3) 	[trans]		for PDF transparency
		%		4)	[notes=hide/show/only]

		%	Text and Math Font
		% 		1)	[sans]
		% 		2)	[sefif]
		%		3) 	[mathsans]
		%		4)	[mathserif]


		%	---------------------------------------------------------	
		%	슬라이드 크기 설정 ( 128mm X 96mm )
		%	---------------------------------------------------------	
%			\setbeamersize{text margin left=2mm}
%			\setbeamersize{text margin right=2mm}

	%	========================================================== 	Package
		\usepackage{kotex}						% 한글 사용
		\usepackage{amssymb,amsfonts,amsmath}	% 수학 수식 사용
		\usepackage{color}					%
		\usepackage{colortbl}					%


	%		========================================================= 	note 옵션인 
	%			\setbeameroption{show only notes}
		

	%		========================================================= 	Theme

		%	---------------------------------------------------------	
		%	전체 테마
		%	---------------------------------------------------------	
		%	테마 명명의 관례 : 도시 이름
%			\usetheme{default}			%
%			\usetheme{Madrid}    		%
%			\usetheme{CambridgeUS}    	% -red, no navigation bar
%			\usetheme{Antibes}			% -blueish, tree-like navigation bar

		%	----------------- table of contents in sidebar
			\usetheme{Berkeley}		% -blueish, table of contents in sidebar
									% 개인적으로 마음에 듬

%			\usetheme{Marburg}			% - sidebar on the right
%			\usetheme{Hannover}		% 왼쪽에 마크
%			\usetheme{Berlin}			% - navigation bar in the headline
%			\usetheme{Szeged}			% - navigation bar in the headline, horizontal lines
%			\usetheme{Malmoe}			% - section/subsection in the headline

%			\usetheme{Singapore}
%			\usetheme{Amsterdam}

		%	---------------------------------------------------------	
		%	색 테마
		%	---------------------------------------------------------	
%			\usecolortheme{albatross}	% 바탕 파란
%			\usecolortheme{crane}		% 바탕 흰색
%			\usecolortheme{beetle}		% 바탕 회색
%			\usecolortheme{dove}		% 전체적으로 흰색
%			\usecolortheme{fly}		% 전체적으로 회색
%			\usecolortheme{seagull}	% 휜색
%			\usecolortheme{wolverine}	& 제목이 노란색
%			\usecolortheme{beaver}

		%	---------------------------------------------------------	
		%	Inner Color Theme 			내부 색 테마 ( 블록의 색 )
		%	---------------------------------------------------------	

%			\usecolortheme{rose}		% 흰색
%			\usecolortheme{lily}		% 색 안 칠한다
%			\usecolortheme{orchid} 	% 진하게

		%	---------------------------------------------------------	
		%	Outter Color Theme 		외부 색 테마 ( 머리말, 고리말, 사이드바 )
		%	---------------------------------------------------------	

%			\usecolortheme{whale}		% 진하다
%			\usecolortheme{dolphin}	% 중간
%			\usecolortheme{seahorse}	% 연하다

		%	---------------------------------------------------------	
		%	Font Theme 				폰트 테마
		%	---------------------------------------------------------	
%			\usfonttheme{default}		
			\usefonttheme{serif}			
%			\usefonttheme{structurebold}			
%			\usefonttheme{structureitalicserif}			
%			\usefonttheme{structuresmallcapsserif}			



		%	---------------------------------------------------------	
		%	Inner Theme 				
		%	---------------------------------------------------------	

%			\useinnertheme{default}
			\useinnertheme{circles}		% 원문자			
%			\useinnertheme{rectangles}		% 사각문자			
%			\useinnertheme{rounded}			% 깨어짐
%			\useinnertheme{inmargin}			




		%	---------------------------------------------------------	
		%	이동 단추 삭제
		%	---------------------------------------------------------	
%			\setbeamertemplate{navigation symbols}{}

		%	---------------------------------------------------------	
		%	문서 정보 표시 꼬리말 적용
		%	---------------------------------------------------------	
%			\useoutertheme{infolines}


			
	%	---------------------------------------------------------- 	배경이미지 지정
%			\pgfdeclareimage[width=\paperwidth,height=\paperheight]{bgimage}{./fig/Chrysanthemum.jpg}
%			\setbeamertemplate{background canvas}{\pgfuseimage{bgimage}}

		%	---------------------------------------------------------	
		% 	본문 글꼴색 지정
		%	---------------------------------------------------------	
%			\setbeamercolor{normal text}{fg=purple}
%			\setbeamercolor{normal text}{fg=red!80}	% 숫자는 투명도 표시


		%	---------------------------------------------------------	
		%	itemize 모양 설정
		%	---------------------------------------------------------	
%			\setbeamertemplate{items}[ball]
%			\setbeamertemplate{items}[circle]
%			\setbeamertemplate{items}[rectangle]






		\setbeamercovered{dynamic}





		% --------------------------------- 	문서 기본 사항 설정
		\setcounter{secnumdepth}{5} 		% 문단 번호 깊이
		\setcounter{tocdepth}{5} 			% 문단 번호 깊이




% ------------------------------------------------------------------------------
% Begin document (Content goes below)
% ------------------------------------------------------------------------------
	\begin{document}
	

			\title{소림사 고려전통사경}

			\author{김대희}

			\date{2020년 01월 07일}


	%	==========================================================
	%		개정 이력
	%	----------------------------------------------------------
	%	2020.01.07 첫 작성
	%	----------------------------------------------------------
	%	
	%	----------------------------------------------------------


	%	==========================================================
	%
	%	----------------------------------------------------------
		\begin{frame}[plain]
		\titlepage
		\end{frame}



%		\begin{frame} [plain]{목차}
		\begin{frame} {목차}
		\tableofcontents
		\end{frame}
		

	%	========================================================== 	일일 업무 ==========================================================
	%		Frame
	%	----------------------------------------------------------
		\part{기본 현황}
		\frame{\partpage}


		\begin{frame} [plain]{목차}
		\tableofcontents
		\end{frame}
		

		
				
		
	%	 ---------------------------------------------------------- 기본개요
	%	 Frame
	%	 ----------------------------------------------------------
		\section{기본 개요}
		\frame [plain] {\sectionpage}
		

		\begin{frame} [t,plain]
			\begin{block} {기본 개요}
			\begin{itemize}
				\item 강좌명 : 
				\item 시간 : 매주 화요일
				\item 장소 : 소림사 소법당
				\item 연락처 : 010 7595 9475
			\end{itemize}
			
			\end{block}
		\end{frame}
		



	%	 ---------------------------------------------------------- 고려사경원
	%	 Frame
	%	 ----------------------------------------------------------
		\section{목림선생}
		\frame [plain] {\sectionpage}

	%	 ----------------------------------------------------------
		\begin{frame} [t,plain]
			\begin{block} {이신 목림 김기태 선생}
			\begin{itemize}
				\item 전번
				\item 팩스
				\item 주소 부산광역시 남구 우암동 265-3번지(48480)
				\item 우편번호 48480
			\end{itemize}
			\end{block}
		\end{frame}

		

% ------------------------------------------------------------------------------ ------------------------------------------------------------------------------ ------------------------------------------------------------------------------
% End document
% ------------------------------------------------------------------------------ ------------------------------------------------------------------------------ ------------------------------------------------------------------------------
\end{document}


	%	----------------------------------------------------------
	%		Frame
	%	----------------------------------------------------------
		\begin{frame} [c]
%		\begin{frame} [b]
%		\begin{frame} [t]
		\frametitle{감리 보고서}
		\end{frame}						

